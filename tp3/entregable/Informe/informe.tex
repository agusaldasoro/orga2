\documentclass[a4paper]{article}
\usepackage[spanish]{babel}
\usepackage[utf8]{inputenc}
\usepackage{charter}   % tipografia
\usepackage{graphicx}
%\usepackage{makeidx}
\usepackage{paralist} %itemize inline

%\usepackage{float}
%\usepackage{amsmath, amsthm, amssymb}
%\usepackage{amsfonts}
%\usepackage{sectsty}
%\usepackage{charter}
%\usepackage{wrapfig}
%\usepackage{listings}
%\lstset{language=C}


\input{codesnippet}
\input{page.layout}
% \setcounter{secnumdepth}{2}
\usepackage{underscore}
\usepackage{caratula}
\usepackage{url}


% ******************************************************** %
%              TEMPLATE DE INFORME ORGA2 v0.1              %
% ******************************************************** %
% ******************************************************** %
%                                                          %
% ALGUNOS PAQUETES REQUERIDOS (EN UBUNTU):                 %
% ========================================
%                                                          %
% texlive-latex-base                                       %
% texlive-latex-recommended                                %
% texlive-fonts-recommended                                %
% texlive-latex-extra?                                     %
% texlive-lang-spanish (en ubuntu 13.10)                   %
% ******************************************************** %



\begin{document}


\thispagestyle{empty}
\materia{Organización del Computador II}
\submateria{Segundo Cuatrimestre de 2014}
\titulo{Trabajo Práctico III}
\subtitulo{System Programming - Zombi defense}
\integrante{Aldasoro Agustina}{86/13}{agusaldasoro@gmail.com}
\integrante{Rey Maximiliano}{XXX/XX}{mail}
\integrante{Tirabasso Ignacio}{XXX/XX}{mail}


\maketitle
\newpage

%\thispagestyle{empty}
%\vfill
%\begin{abstract}
%En el presente trabajo se describe la problemática de ...
%\end{abstract}

%\thispagestyle{empty}
%\vspace{3cm}
%\tableofcontents
\newpage


%\normalsize
\newpage
\section{Ejercio 1}
{\large a)} Armamos los cuatro segmentos de la GDT, llam\'andolos: 
\begin{codesnippet}
\begin{verbatim}
    [GDT_IDX_CODE_0] = (gdt_entry) ; 
    [GDT_IDX_CODE_3] = (gdt_entry) ; 
    [GDT_IDX_DATA_0] = (gdt_entry) ;  
    [GDT_IDX_DATA_3] = (gdt_entry) ; 
\end{verbatim}
\end{codesnippet}
    
A los cuatro les seteamos el mismo \emph{l\'imite}: 0x26EFF = ... , y la misma \emph{base} en 0. El \emph{segment type} var\'ia depende el segmento: CODE_0: 0x0A= ... , CODE_3: 0x0F= ... , DATA_0 y DATA_3: 0x02= ... . El \emph{Descriptor type} va en todos para system, por lo tanto es 0. El \emph{Descriptor privilege level} coincide con el nombre del descriptor (0 para CODE_0 y DATA_0; 3 para CODE_3 y DATA_3). El bit de \emph{Present} va para todos en 1 y los bit de \emph{Available for use by system software} y \emph{l} van para todos en 0. El bit de \emph{Default operation size} va para todos en 1 porque es un c\'odigo de 32bits. El bit de \emph{Granularity} va para todos en 1.\\
    


{\large b)} Se adjunta el c\'odigo necesario para pasar a modo protegido y setear la pila del kernel en
la direcci\'on 0x27000.


\begin{codesnippet}
\begin{verbatim}
    ; Deshabilitar interrupciones
    cli
  
    ; Habilitar A20
    call habilitar_A20
	
    ; Cargar la GDT
    lgdt [GDT_DESC]

    ; Setear el bit PE del registro CR0
    mov eax,cr0
    or eax,1
    mov cr0,eax

    jmp 0x50:modo_protegido
\end{verbatim}
\end{codesnippet}
\begin{codesnippet}
\begin{verbatim}
BITS 32
modo_protegido:

    ; Establecer selectores de segmentos
    xor eax, eax
    mov ax, 0x40
     
    mov es, ax
    mov ds, ax
    mov ss, ax    
    mov gs, ax

    mov ax, 0x60 
    mov fs, ax
    
    ; Establecer la base de la pila
    mov ebp, 0x27000
\end{verbatim}
\end{codesnippet}

{\large c)} Segmento adicional que describe el \'area de la pantalla en memoria que puede
ser utilizado solo por el kernel:
\begin{codesnippet}
\begin{verbatim}
    ; Cambiar modo de video a 80 X 50
    mov ax, 0003h
    int 10h ; set mode 03h
    xor bx, bx
    mov ax, 1112h
    int 10h ; load 8x8 font
\end{verbatim}
\end{codesnippet}


{\large d)} La siguiente es la rutina que se encarga de limpiar la pantalla y pintar el \'area del mapa
un fondo de color verde, junto con las dos barras laterales para cada uno de
los jugadores (rojo y azul).
\begin{codesnippet}
\begin{verbatim}
    ; Inicializar pantalla
    call clear_screen
    call print_map
\end{verbatim}
\end{codesnippet}
\begin{codesnippet}
\begin{verbatim}
void clear_screen() {
    int size = VIDEO_COLS * VIDEO_FILS;    
    ca (*p) = (ca (*)) VIDEO; // magia
    int i = 0;
    ca empty;
    empty.c = 0;
    empty.a = getFormat(C_FG_BLACK, 0, C_BG_BLACK, 0);
    while(i < size) {
        p[i] = empty;
        i++;
    }
}
\end{verbatim}
\end{codesnippet}

\begin{codesnippet}
\begin{verbatim}
void print_map() {
    int cols = VIDEO_COLS;
    int rows = VIDEO_FILS;

    ca (*screen)[VIDEO_COLS] = (ca (*)[VIDEO_COLS]) VIDEO;

    ca red;
    red.c = 0;
    red.a = getFormat(C_FG_RED, 0, C_BG_RED, 0);
    ca blue;
    blue.c = 0;
    blue.a = getFormat(C_FG_BLUE, 0, C_BG_BLUE, 0);
    ca green;
    green.c = 0;
    green.a = getFormat(C_FG_GREEN, 0, C_BG_GREEN, 0);
    ca black;
    black.c = 0;
    black.a = getFormat(C_FG_BLACK, 0, C_BG_BLACK, 0);

    int y,x;

    clear_screen();
    
    for(y = 0; y < rows; y++) {
        for(x = 0; x < cols; x++) {
            if (y >= rows-5) {
                screen[y][x] = black;
            } else if (x == cols-1) {
                screen[y][x] = blue;
            } else if (x == 0) {
                screen[y][x] = red;
            } else {
                screen[y][x] = green;
            }
        }
    }
    for(y = rows-5; y < rows; y++) {
        for(x = 35; x < 40; x++) {
            screen[y][x] = red;
        }
    }

    for(y = rows-5; y < rows; y++) {
        for(x = 40; x < 45; x++) {
            screen[y][x] = blue;
        }
    }
}
\end{verbatim}
\end{codesnippet}

\newpage
\section{Ejercio 2}
{\large a)} Completar las entradas necesarias en la IDT para asociar diferentes rutinas a todas las
excepciones del procesador. Cada rutina de excepci\'on debe indicar en pantalla qu\'e problema
se produjo e interrumpir la ejecuci\'on. Posteriormente se modificar\'an estas rutinas
para que se continue la ejecuci\'on, resolviendo el problema y desalojando a la tarea que lo
produjo.\\

{\large b)} Hacer lo necesario para que el procesador utilice la IDT creada anteriormente. Generar
una excepci\'on para probarla.

\newpage
\section{Ejercio 3}
{\large a)} Escribir una rutina que se encargue de limpiar el buffer de video y pintarlo como indica
la figura 9. Tener en cuenta que deben ser escritos de forma gen\'erica para posteriormente
ser completados con informaci\'on del sistema. Adem\'as considerar estas im\'agenes como
sugerencias, ya que pueden ser modificadas a gusto seg\'un cada grupo mostrando siempre
la misma informaci\'on.\\

{\large b)} Escribir las rutinas encargadas de inicializar el directorio y tablas de p\'aginas para el kernel
(mmu inicializar dir kernel). Se debe generar un directorio de p\'aginas que mapee,
usando identity mapping, las direcciones 0x00000000 a 0x003FFFFF, como ilustra la figura
6. Adem\'as, esta funci\'on debe inicializar el directorio de p\'aginas en la direcci\'on 0x27000
y las tablas de p\'aginas seg\'un muestra la figura 1.\\

{\large c)} Completar el c\'odigo necesario para activar paginaci\'on.\\

{\large d)} Escribir una rutina que imprima el nombre del grupo en pantalla. Debe estar ubicado en
la primer l\'inea de la pantalla alineado a derecha.
4.4.

\newpage
\section{Ejercio 4}
{\large a)} Escribir una rutina (inicializar mmu) que se encargue de inicializar las estructuras necesarias
para administrar la memoria en el area libre (un contador de paginas libres).\\

{\large b)} Escribir una rutina (mmu inicializar dir zombi) encargada de inicializar un directorio
de p\'aginas y tablas de p\'aginas para una tarea, respetando la figura 6. La rutina debe copiar
el c\'odigo de la tarea a su \'area asignada, es decir la posici\'on indicada por el jugador dentro
de el mapa y mapear dichas p\'aginas a partir de la direcci\'on virtual 0x08000000(128MB).
Recordar que los zombis comienzan en la segunda columna de el mapa y en la fila correspondiente
a la posici\'on donde esta el jugador. Sugerencia: agregar a esta funci\'on todos los
par\'amentros que considere necesarios.\\

{\large c)} Escribir dos rutinas encargadas de mapear y desmapear p\'aginas de memoria.

I- mmu mapear p\'agina(unsigned int virtual, unsigned int cr3, unsigned int f\'isica)
Permite mapear la p\'agina f\'isica correspondiente a f\'isica en la direcci\'on virtual
virtual utilizando cr3.

II- mmu unmapear pagina(unsigned int virtual, unsigned int cr3)
Borra el mapeo creado en la direcci\'on virtual virtual utilizando cr3. \\

{\large d)} Construir un mapa de memoria para tareas e intercambiarlo con el del kernel, luego cambiar
el color del fondo del primer caracter de la pantalla y volver a la normalidad. Este item
no debe estar implementado en la soluci\'on final.


\newpage
\section{Ejercio 5}

{\large a)} Completar las entradas necesarias en la IDT para asociar una rutina a la interrupci\'on del
reloj, otra a la interrupci\'on de teclado y por \'ultimo una a la interrupci\'on de software 0x66. \\

{\large b)} Escribir la rutina asociada a la interrupci\'on del reloj, para que por cada tick llame a la
funci\'on screen pr\'oximo reloj. La misma se encarga de mostrar cada vez que se llame, la
animaci\'on de un cursor rotando en la esquina inferior derecha de la pantalla. La funci\'on
pr\'oximo reloj est\'a definida en isr.asm. \\

{\large c)} Escribir la rutina asociada a la interrupci\'on de teclado de forma que si se presiona cualquiera
de las teclas a utilizar en el juego, se presente la misma en la esquina superior
derecha de la pantalla. \\

{\large d)} Escribir la rutina asociada a la interrupci\'on 0x66 para que modifique el valor de eax por
0x42. Posteriormente este comportamiento va a ser modificado para atender el servicio del
sistema.


\newpage
\section{Ejercio 6}
{\large a)} Definir las entradas en la GDT que considere necesarias para ser usadas como descriptores
de TSS. Minimamente, una para ser utilizada por la tarea inicial y otra para la tarea
Idle.\\

{\large b)} Completar la entrada de la TSS de la tarea Idle con la informaci\'on de la tarea Idle. Esta
informaci\'on se encuentra en el archivo TSS.C. La tarea Idle se encuentra en la direcci\'on
0x00016000. La pila se alojar\'a en la misma direcci\'on que la pila del kernel y ser\'a mapeada
con identity mapping. Esta tarea ocupa 1 pagina de 4KB y debe ser mapeada con identity
mapping. Adem\'as la misma debe compartir el mismo CR3 que el kernel. \\

{\large c)} Construir una funci\'on que complete una TSS libre con los datos correspondientes a una
tarea (zombi). El c\'odigo de las tareas se encuentra a partir de la direcci\'on 0x00010000
ocupando una pagina de 4kb cada una seg\'un indica la figura 1. Para la direcci\'on de la
pila se debe utilizar el mismo espacio de la tarea, la misma crecer\'a desde la base de la tarea. Para el mapa de memoria se debe construir uno nuevo utilizando la funci\'on
mmu inicializar dir zombi. Adem\'as, tener en cuenta que cada tarea utilizar\'a una pila
distinta de nivel 0, para esto se debe pedir una nueva pagina libre a tal fin. \\

d) Completar la entrada de la GDT correspondiente a la tarea inicial.\\

e) Completar la entrada de la GDT correspondiente a la tarea Idle.\\

f) Escribir el c\'odigo necesario para ejecutar la tarea Idle, es decir, saltar intercambiando las
TSS, entre la tarea inicial y la tarea Idle.


\newpage
\section{Ejercio 7}
{\large a)} Construir una funci\'on para inicializar las estructuras de datos del scheduler.\\

{\large b)} Crear la funci\'on sched pr\'oximo indice() que devuelve el \'indice en la GDT de la pr\'oxima
tarea a ser ejecutada. Construir la rutina de forma devuelva una tarea de cada jugador
por vez seg\'un se explica en la secci\'on 3.2\\

{\large c)} Modificar la rutina de la interrupci\'on 0x66, para que implemente el servicio mover seg\'un
se indica en la secci\'on 3.1.1.\\

{\large d)} Modificar el c\'odigo necesario para que se realice el intercambio de tareas por cada ciclo de
reloj. El intercambio se realizar\'a seg\'un indique la funci\'on sched proximo \'indice().\\

{\large e)} Modificar las rutinas de excepciones del procesador para que desalojen a la tarea que
estaba corriendo y corran la pr\'oxima.\\

{\large f)} Implementar el mecanismo de debugging explicado en la secci\'on 3.4 que indicar\'a en pantalla
la raz\'on del desalojo de una tarea.


\newpage
\section{Ejercio 8}
{\large a)} Crear un conjunto de 3 tareas zombis (Guerrero, Mago y Clerigo). Los mismos deber\'an respetar
las restricciones del trabajo pr\'actico, ya que de no hacerlo no podr\'an ser ejecutados
en el sistema implementado por la c\'atedra.

Deben cumplir:

No ocupar m\'as de 4 kb cada uno (tener en cuenta la pila).

Tener como punto de entrada la direcci\'on cero.

Estar compilado para correr desde la direcci\'on 0x08000000.

Utilizar el unico servicio del sistema (mover).

Explicar en pocas palabras qu\'e estrategia utiliza cada uno de los zombis, o en su conjunto
en t\'erminos de defensa y ataque.\\

{\large b)} Si consideran que sus tareas pueden hacer algo mas que completar el primer item de este
ejercicio, y tienen a un audaz campion que se atreva a enfrentarse en el campo de batalla
zombi, entonces pueden enviar el binario de sus tareas a la lista de docentes indicando
los siguientes datos:

Nombre del campion (Alumno de la materia que se presente como jugador)

Nombre de cada uno de las tareas zombi

Estrategia de alimentaci\'on de los zombis (es decir, como se comer\'an los cerebros de
las otras tareas)

Se realizar\'a una competencia a fin de cuatrimestre con premios en/de chocolate para los
primeros puestos. \\

{\large c)} Pelicula y Video Juego favorito sobre Zombis.


\newpage
\section{Conclusiones y trabajo futuro}


\end{document}






%%%%%%%%%%%%%%%%%%%%%%%%%%%%%%%%%%%%%%%%%%%%%%%%%%%%%%%%%%%%%%%%%%%


\section{Objetivos generales}

El objetivo de este Trabajo Práctico es ...


\section{Contexto}

\begin{figure}
  \begin{center}
	\includegraphics[scale=0.66]{imagenes/logouba.jpg}
	\caption{Descripcion de la figura}
	\label{nombreparareferenciar}
  \end{center}
\end{figure}


\paragraph{\textbf{Titulo del parrafo} } Bla bla bla bla.
Esto se muestra en la figura~\ref{nombreparareferenciar}.



\begin{codesnippet}
\begin{verbatim}

struct Pepe {

    ...

};

\end{verbatim}
\end{codesnippet}


%\section{Enunciado y solucion} 
%\subsection{Mediciones}

Realizar una medición de performance \emph{rigurosa} es más difícil de lo 
que parece. 
En este experimento deberá realizar distintas mediciones de performance 
para verificar que sean buenas mediciones.

En un sistema ``ideal'' el proceso medido corre solo, sin ninguna 
interferencia de agentes externos. 
Sin embargo, una PC no es un sistema ideal. 
Nuestro proceso corre junto con decenas de otros, tanto de usuarios como 
del sistema operativo que compiten por el uso de la CPU. 
Esto implica que al realizar mediciones aparezcan ``ruidos'' o 
``interferencias'' que distorsionen los resultados.

El primer paso para tener una idea de si la medición es buena o no, 
es tomar varias muestras. 
Es decir, repetir la misma medición varias veces.
Luego de eso, es conveniente descartar los outliers
\footnote{en español, valor atípico: \url{http://es.wikipedia.org/wiki/Valor_atípico}}, 
que son los valores que más se alejan del promedio. 
Con los valores de las mediciones resultantes se puede calcular el promedio 
y también la varianza, que es algo similar el promedio de las distancias al 
promedio\footnote{en realidad, elevadas al cuadrado en vez de tomar el módulo}.

Las fórmulas para calcular el promedio $\mu$ y la varianza $\sigma^2$ son

$$
\mu = \frac{1}{n}\sum_{i=1}^{n} x_i \qquad \sigma^2 = \frac{\displaystyle\sum_{i=1}^{n}(x_i - \mu)^2} {n}
$$
%------------------------------------------

\subsection{Filtro \textit{cropflip}}

Programar el filtro \textit{cropflip} en lenguaje C y luego en ASM haciendo 
uso de las instrucciones vectoriales (\textbf{SSE}).

% ******************************************************************************
\vspace*{0.3cm} \noindent
\textbf{Experimento 1.1 - análisis el código generado}

En este experimento vamos a utilizar la herramienta \verb|objdump| para 
verificar como el compilador de C deja ensamblado el código C.

Ejecutar 
\begin{codesnippet}
\begin{verbatim}
objdump -Mintel -D cropflip_c.o
\end{verbatim}
\end{codesnippet}

¿Cómo es el código generado? 
Indicar
\begin{inparaenum}[\itshape a\upshape)]
    \item Por qué cree que hay otras funciones además de \verb|cropflip_c|
    \item Cómo se manipulan las variables locales
    \item Si le parece que ese código generado podría optimizarse
\end{inparaenum}

% ******************************************************************************
%\newpage
\vspace*{0.3cm} \noindent
\textbf{Experimento 1.2 - optimizaciones del compilador}

Compile el código de C con flags de optimización. Por ejemplo, pasando el flag 
\verb|-O1|\footnote{agregando este flag a \texttt{CCFLAGS64} en el makefile}. 
Indicar
\begin{inparaenum}
    \item Qué optimizaciones observa que realizó el compilador
    \item Qué otros flags de optimización brinda el compilador
    \item Los nombres de tres optimizaciones que realizan los compiladores.
\end{inparaenum}

%----------------------------------------------
\vspace*{0.3cm} \noindent
\textbf{Experimento 1.3 - calidad de las mediciones}

\begin{enumerate}
    \item Medir el tiempo de ejecución de cropflip 10 veces. Calcular el promedio y la varianza. Consideraremos outliers a los 2 mayores tiempos de ejecución de la medicion y también a los 2 menores, por lo que los descartaremos. Recalcular el promedio y la varianza después de hacer este descarte. Realizar un gráfico que presente estos dos últimos items.\\
\\
Luego de ejecutar 10 veces el filtro Cropflip obtuvimos los siguientes resultados: \\
\\
    	\begin{tabular}[c]{|c|c|c|}
	\hline
		\textbf{ASM} & \textbf{C}\\
		\hline
70.925 &	1.152.187\\
		\hline
70.521 &	1.151.544\\
		\hline
32.859 &	761.937\\
		\hline
43.720 &	649.248\\
		\hline
64.236 &	1.152.847\\
		\hline
70.793 &	1.153.061\\
		\hline
71.271 &	1.152.765\\
		\hline
44.616 &	1.152.798\\
		\hline
56.124 &	725.420\\
		\hline
71.775 &	1.155.718\\
		\hline
	\textbf{Esperanza}	\\
		\hline
59.684 & 1.020.752,5	\\
		\hline
		\textbf{Desvío standard}	\\
		\hline
13.720,5329 & 203.626,443	\\
		\hline
	\end{tabular}\\\\
	El cuadro denota la cantidad de ciclos de clock utilizada por cada ejecuci\'on del programa. \\
	\\
	Luego de eliminar los dos valores m\'as altos y los dos valores m\'as bajos, recalculamos obteniendo los siguientes datos: \\
	\textbf{Esperanza}: 62.869,166
 (ASM) y 1.087.346,333(C)\\
	\textbf{Desvío standard}:	9.719,205 (ASM) y 145.528,202(C)\\
	Se puede ver que al eliminar los outliers, la esperanza comienza a converger a su valor esperado y el desvío standard disminuye. \\
\\
\newpage
\begin{figure}
  \begin{center}
	\includegraphics[width=0.7\textwidth]{imagenes/13/asm1.jpg}
	\caption{Assembler}
%	\label{nombreparareferenciar}
	\includegraphics[width=0.7\textwidth]{imagenes/13/C1.jpg}
	\caption{C}
  \end{center}
\end{figure}
\newpage
\indent Siendo el Caso 1 las mediciones de esperanza y Desvío standard para todos los casos de test y el Caso 2 las mediciones sin tener en cuenta los cuatro outliers.\\
    \item Implementar un programa en C que no haga más que ciclar infinitamente sumando 1 a una variable. Lanzar este programa tantas veces como \emph{cores lógicos} tenga su procesador. Medir otras 10 veces mientras estos programas corren de fondo. Realizar los mismos casos de experimentaci\'on que en el ejercicio anterior.\\
\end{enumerate}
Los resultados obtenidos en esta experimentaci\'on fueron menores que los anteriores: \\
\\       
        \begin{tabular}[c]{|c|c|c|}
	\hline
		\textbf{ASM} & \textbf{C}\\
		\hline
33.585 &	542.928 \\
\hline
33.798 &	544.155 \\
\hline
33.402 &	544.857 \\
\hline
33.228 &	543.687 \\
\hline
33.159 &	543.252 \\
\hline
33.441 &	543.324 \\
\hline
34.089 &	544.224 \\ 
\hline
33.768 &	760.359 \\ 
\hline
34.563 &	542.448 \\
\hline
34.473 &	542.982 \\
\hline
		\textbf{Esperanza}	\\
		\hline
33.750,6 & 565.221,6	\\		
		\hline
		\textbf{Desvío standard}	\\
		\hline
465,49 & 65.049,34\\
		\hline
	\end{tabular}\\\\
	Luego de eliminar los dos valores m\'as altos y los dos valores m\'as bajos, recalculamos obteniendo los siguientes datos: \\
	\textbf{Esperanza}: 33.680,5 (ASM) y 543.604(C)\\
	\textbf{Desvío standard}:	235,364 (ASM) y 462,615(C)\\
	Ac\'a tambi\'en se puede apreciar que al eliminar los outliers, el Desvío standard disminuye su valor. \\
\newpage
\begin{figure}
  \begin{center}
	\includegraphics[width=0.7\textwidth]{imagenes/13/asm2.jpg}
	\caption{Assembler}
%	\label{nombreparareferenciar}
	\includegraphics[width=0.7\textwidth]{imagenes/13/C2.jpg}
	\caption{C}
  \end{center}
\end{figure}
\newpage
\indent Siendo el Caso 1 las mediciones de esperanza y Desvío standard para todos los casos de test y el Caso 2 las mediciones sin tener en cuenta los cuatro outliers. Se puede observar que las mediciones mejoran con la ejecuci\'on del ciclo infinito de fondo, esto se debe a que fue ejecutado en una computadora con un procesador i5. \\
\\
\textit{A partir de aquí todos los experimentos de mediciones deberán hacerse igual 
que en el presente ejercicio: tomando 10 mediciones, luego descartando 
outliers y finalmente calculando promedio y Desvío standard.}\\
\\
Decidimos: \\
Realizar 12000 mediciones por experimento, eliminando los primeros mil casos que hayan llevado menos ciclos de clock y los mil casos que hayan llevado la mayor cantidad de ciclos de clock. \\
Lo determinamos de esta manera, ya que dejar dos mediciones afuera, como dice el enunciado, no tiene influencia en los c\'alculos de la esperanza y la varianza para muestras tan grandes. \\
Luego de experimentar distintas cantidades de casos de testeo, notamos que elegir 7000 valores pertenecientes a la franja del medio de los 12000 es una soluci\'on lo suficientemente establo, por lo cual es la que llevamos a cabo. \\
\\
% ******************************************************************************
%\newpage
\noindent\textbf{Experimento 1.4 - secuencial vs. vectorial}

En este experimento deberá realizar una medición de las diferencias de 
performance entre las versiones de C y ASM (el primero con -O0, -O1, -O2 y -O3) 
y graficar los resultados. \\
\\
El siguiente gr\'afico indica la esperanza de la cantidad de ciclos de clock que toma ejecutar el filtro Cropflip con los par\'ametros 404 404 4 4 en ASM y en C variando los flags de o0 a o3. \\
Reflejando las siguientes magnitudes: \\
\\
 \begin{tabular}[c]{|c|c|c|}
	\hline
		 & Esperanza & Desv\'io Standard\\
		\hline
C -o0 & 6.761.044,5 & 131,463 \\
\hline
C -o1 & 1.300.109 & 53,447 \\
\hline
C -o2 & 1.227.102,5 & 50,143\\
\hline
C -o3 & 266.688 & 0,287 \\
\hline
ASM & 362.232 & 2,079\\
\hline
	\end{tabular}\\\\
\\
Se puede observar que la varianza es casi despreciable considerando el valor de la esperanza. \\
\newpage
\begin{figure}
  \begin{center}
	\includegraphics[width=0.7\textwidth]{imagenes/14.jpg}
  \end{center}
\end{figure}
\newpage

% ******************************************************************************
\vspace*{0.3cm} \noindent
\textbf{Experimento 1.5 - cpu vs. bus de memoria}

Se desea conocer cual es el mayor limitante a la
performance de este filtro en su versión ASM.

¿Cuál es el factor que limita la performance en este caso?
En caso de que el limitante fuera la intensidad de cómputo, entonces 
podrían agregarse instrucciones que realicen accesos a memoria extra y la
performance casi no debería sufrir. 
La inversa puede aplicarse, si el limitante es la cantidad de accesos a memoria.
\footnote{también podría pasar que estén más bien balanceados y que agregar
cualquier tipo de instrucción afecte sensiblemente la performance}
	
Realizar un experimento, agregando 4, 8 y 16 instrucciones aritméticas 
(por ej \verb|add rax, rbx|) analizando como varía el tiempo de ejecución.
Hacer lo mismo ahora con instrucciones de acceso a memoria, haciendo 
mitad lecturas y mitad escrituras (por ejemplo, agregando dos 
\verb|mov rax, [rsp]| y dos \verb|mov [rsp+8], rax|).\footnote{Notar que en el caso de acceder a \texttt{[rbp]} o \texttt{[rsp+8]} probablemente haya siempre hits en la cache, por lo que la medición no será de buena calidad. Si se le ocurre la manera, realizar accesos a otras direcciones alternativas.}
	
Realizar un único gráfico que compare:
\begin{inparaenum}
    \item La versión original
    \item Las versiones con más instrucciones aritméticas
    \item Las versiones com más accesos a memoria
\end{inparaenum}

Acompañar al gráfico con una tabla que indique los valores graficados.  
  
% ------------------------------------------------------------------------------

\subsection{Filtro \textit{Sierpinski}}

Programar el filtro \textit{Sierpinski} en lenguaje C y en en ASM haciendo 
uso de las instrucciones vectoriales (\textbf{SSE}).

% ******************************************************************************
\vspace*{0.3cm} \noindent
\textbf{Experimento 2.1 - secuencial vs. vectorial}

Analizar cuales son las diferencias de performace entre las versiones de C 
y ASM de este filtro, de igual modo que para el experimento 1.4. \\
\\
El siguiente gr\'afico indica la esperanza de la cantidad de ciclos de clock que toma ejecutar el filtro Sierpinski en ASM y en C variando los flags de o0 a o3. \\
Reflejando las siguientes magnitudes: \\
\\
 \begin{tabular}[c]{|c|c|c|}
	\hline
		 & Esperanza & Desv\'io Standard\\
		\hline
C -o0 & 28.801.586,5 & 78,016 \\
\hline
C -o1 & 21.395.732 & 78,447 \\
\hline
C -o2 & 14.231.758 & 116,445 \\
\hline
C -o3 & 14.229.650 & 116,022 \\
\hline
ASM & 3.661.626 & 115,583 \\
\hline
	\end{tabular}\\\\
\\
Aqu\'i tambi\'en el valor del desv\'io standard es despreciable. \\
\newpage
\begin{figure}
  \begin{center}
	\includegraphics[width=0.7\textwidth]{imagenes/21.jpg}
  \end{center}
\end{figure}
\newpage

% ******************************************************************************
\vspace*{0.3cm} \noindent
\textbf{Experimento 2.1 - cpu vs. bus de memoria}

¿Cuál es el factor que limita la performance en este filtro?
Repetir el experimento 1.5 para este filtro.

\subsection{Filtro \textit{Bandas}}

Programar el filtro \textit{Bandas} en lenguaje C y en en ASM haciendo uso de 
las instrucciones vectoriales (\textbf{SSE}).

% ******************************************************************************
\vspace*{0.3cm} \noindent
\textbf{Experimento 3.1 - saltos condicionales}

Se desea conocer que tanto impactan los saltos condicionales en el código 
de filtro Bandas con \verb|-O1| (la versión en C).\\
Para poder medir esto de manera aproximada, remover el código
que detecta a que banda pertenece cada pixel, dejando
sólo una banda.
Por más que la imagen resultante no sea correcta, será posible tomar una
medida aproximada del impacto de los saltos condicionales.
Analizar como varía la performance. 

% ******************************************************************************
\vspace*{0.3cm} \noindent
\textbf{Experimento 3.2 - secuencial vs. vectorial}

Repetir el experimento 1.4 para este filtro. \\
\\
El siguiente gr\'afico indica la esperanza de la cantidad de ciclos de clock que toma ejecutar el filtro Bandas en ASM y en C variando los flags de o0 a o3. \\
Reflejando las siguientes magnitudes: \\
\\
 \begin{tabular}[c]{|c|c|c|}
	\hline
		 & Esperanza & Desv\'io Standard\\
		\hline
C -o0 & 17.472.791 & 120,620 \\
\hline
C -o1 & 4.583.340 & 128,514 \\
\hline
C -o2 & 3.259.839 & 117,746 \\
\hline
C -o3 & 3.259.767 & 117,391  \\
\hline
ASM & 3.703.203,5 & 115,659 \\
\hline
	\end{tabular}\\\\
\\
Aqu\'i tambi\'en el valor del desv\'io standard es despreciable. \\
\newpage
\begin{figure}
  \begin{center}
	\includegraphics[width=0.7\textwidth]{imagenes/32.jpg}
  \end{center}
\end{figure}
\newpage

% ------------------------------------------------------------------------------

\subsection{Filtro \textit{Motion Blur}}
Programar el filtro \textit{mblur} en lenguaje C y en ASM haciendo uso de 
las instrucciones \textbf{SSE}.

% ******************************************************************************
\vspace*{0.3cm} \noindent
\textbf{Experimento 4.1}

Repetir el experimento 1.4 para este filtro \\
\\
El siguiente gr\'afico indica la esperanza de la cantidad de ciclos de clock que toma ejecutar el filtro Bandas en ASM y en C variando los flags de o0 a o3. \\
Reflejando las siguientes magnitudes: \\
\\
 \begin{tabular}[c]{|c|c|c|}
	\hline
		 & Esperanza & Desv\'io Standard\\
		\hline
C -o0 & 24.652.732,5 & 117,738 \\
\hline
C -o1 & 11.562.261 & 103,024  \\
\hline
C -o2 & 9.641.346 & 78,159  \\
\hline
C -o3 & 9.639.625 & 78,227 \\
\hline
ASM & 2.993.980,5 & 118,245 \\
\hline
	\end{tabular}\\\\
\\
Aqu\'i tambi\'en el valor del desv\'io standard es despreciable. \\
\newpage
\begin{figure}
  \begin{center}
	\includegraphics[width=0.7\textwidth]{imagenes/41.jpg}
  \end{center}
\end{figure}
\newpage
